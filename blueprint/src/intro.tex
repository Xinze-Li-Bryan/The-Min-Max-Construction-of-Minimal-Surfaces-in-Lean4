\chapter{Introduction}

We formalize the Simon--Smith min--max construction of minimal surfaces in closed $3$-manifolds.

\begin{theorem}[Simon--Smith, Colding--De Lellis]\label{thm:main}
\lean{MinimalSurfaces.simon_smith}
For any saturated set of generalized families of surfaces $\Lambda$, there exists a min--max sequence obtained from $\Lambda$ converging in the varifold sense to a smooth embedded minimal surface of area $m_0(\Lambda)$ (multiplicity allowed).
\uses{def:saturated, def:gen-family, def:m0}
\end{theorem}

This theorem represents the culmination of the min--max theory for minimal surfaces.

\section{The min--max construction in 3--manifolds}

In the following, $M$ denotes a closed 3-dimensional Riemannian manifold, $\mathrm{Diff}_0$ is the identity component of the diffeomorphism group of $M$, and $\mathrm{Is}$ is the set of smooth isotopies.

\begin{definition}[Continuous Family]\label{def:continuous-family}
\lean{MinimalSurfaces.ContinuousFamily}
A family $\{\Sigma_t\}_{t \in [0,1]}$ of surfaces of $M$ is said to be \emph{continuous} if:
\begin{itemize}
\item[(c1)] $\mathcal{H}^2(\Sigma_t)$ is a continuous function of $t$;
\item[(c2)] $\Sigma_t \to \Sigma_{t_0}$ in the Hausdorff topology whenever $t \to t_0$.
\end{itemize}
\uses{}
\end{definition}

\begin{definition}[Generalized Family of Surfaces]\label{def:gen-family}
\lean{MinimalSurfaces.GeneralizedFamilyOfSurfaces}
\leanok
A family $\{\Sigma_t\}_{t \in [0,1]}$ of subsets of $M$ is said to be a \emph{generalized family of surfaces} if there are a finite subset $T$ of $[0, 1]$ and a finite set of points $P$ in $M$ such that:
\begin{itemize}
\item[(1)] Conditions (c1) and (c2) from Definition~\ref{def:continuous-family} hold;
\item[(2)] $\Sigma_t$ is a surface for every $t \notin T$;
\item[(3)] For $t \in T$, $\Sigma_t$ is a surface in $M \setminus P$.
\end{itemize}
\uses{def:continuous-family}
\end{definition}